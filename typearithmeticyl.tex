% !TeX TXS-program:compile = txs:///pdflatex/[--shell-escape]
\documentclass[tikz]{beamer}


\usepackage{minted}
\usepackage[parfill]{parskip}    		
\usepackage{graphicx}				
\usepackage{amssymb}
\usepackage{amsfonts}
\usepackage{amsmath}
\usepackage{bm}
\usepackage{tikz-cd}
\usepackage{enumerate}
\usepackage{xfrac}
\usepackage{hyperref}
\usepackage{graphicx} \graphicspath{ {./} }
\usepackage{xcolor}


\newcommand{\cat}[1]{\bm{ \mathsf{#1} }}
\newcommand{\functor}[3]{#1 : \cat{#2} \to \cat{#3}}
\newcommand{\functordef}{\functor{F}{C}{D}}
\newcommand{\cc}{\cat{C}}
\newcommand{\dd}{\cat{D}}
\newcommand{\ee}{\cat{E}}
\newcommand{\subcat}[2]{\bm{ \mathsf{#1}}_{\bm{ \mathsf{#2}}}}
\newcommand{\op}[1]{#1^{\text{op}}}
\newcommand{\opc}{\op{\cc}}
\newcommand{\opd}{\op{\dd}}
\newcommand{\ope}{\op{\ee}}
\newcommand{\mono}{\rightarrowtail}
\newcommand{\epi}{\twoheadrightarrow}
\newcommand{\one}{\mathbb{1}}
\newcommand{\bg}{\cat{BG}}
\newcommand{\bgg}{\cat{BG'}}
\newcommand{\nt}{\Rightarrow}
\newcommand{\ant}[2]{\alpha : F \nt G} 
\newcommand{\bnt}[2]{\beta : F \nt G} 
\newcommand{\anti}[2]{\alpha : F \cong G} 
\newcommand{\bnti}[2]{\beta : F \cong G} 
\newcommand{\red}[1]{\textcolor{red}{#1}}
\newcommand{\mred}[1]{\textcolor{red}{$#1$}}
\newcommand{\blue}[1]{\textcolor{blue}{#1}}
\newcommand{\mblue}[1]{\textcolor{blue}{$#1$}}

\setbeamertemplate{navigation symbols}{\insertframenumber{}}
\colorlet{shadecolor}{gray!15}
\usepackage[utf8]{inputenc}
\usepackage[english]{babel}
\newcommand{\propnumber}{} % initialize
\newtheorem*{prop}{Proposition \propnumber}
\theoremstyle{definition}
\newtheorem{defn}{Definition}[section]

\AtBeginSection[]{
  \begin{frame}
  \vfill
  \centering
  \begin{beamercolorbox}[sep=8pt,center,shadow=true,rounded=true]{title}
    \usebeamerfont{title}\insertsectionhead\par%
  \end{beamercolorbox}
  \vfill
  \end{frame}
}


\title{Type Arithmetic and the Yoneda Lemma}
\author{Emily Pillmore}

\begin{document}

\frame{\titlepage}

\section[Outline]{}
\frame{\tableofcontents}

\section{Introduction}
\subsection{Who am I?}

\frame
{
	My name is Emily Pillmore.
	
	I am a hopeful mathematician, and a moonlight programmer.
}

\frame
{
	\begin{itemize}
		\item Twitter (\href{https://twitter.com/emi1ypi}{@emi1ypi})
		\item Meetups in NYC: NY Homotopy Type Theory, NY Category Theory, and the NY Haskell User Group. 
		\item Discord: Haskell $\cap$ Dank Memes: \href{https://discord.gg/2x2fYSK}{https://discord.gg/2x2fYSK}. 
		\item Personal: All of my slides and meetup content are hosted at \href{https://github.com/cohomolo-gy}{cohomolo.gy}.
	\end{itemize}
	
}

\frame
{
	I got my start in Scala, with Runar's \textit{Functional Programming in Scala} (the \textcolor{red}{red one} - there is now a \textcolor{blue}{blue one})
}


\frame
{ 
	I now work at a company called \textbf{Kadena}, designing a language and a couple of cool blockchains
	\begin{center}
	
		\includegraphics[scale=0.3]{kadena}
	\end{center}
	
}

\subsection{Why are we here?}

\frame
{ 
	Today, we're going to learn to \textbf{count}. Then we will learn to \textbf{add}, \textbf{multiple}, and even \textbf{take powers}. 
}

\frame
{ 
	...and we will do this with the language of Type Theory, Category Theory, and Constructive Logic. 
}

\frame
{ 
	Many of you know these topics already. Much of it is folklore within the community. But there are new ways of thinking about it that I gaurantee you haven't seen before, and some of it even translates into our work on Profunctor Optics! 
}

\frame
{ 
	If Bartosz is in this audience, everyone stare at him right now. 
}

\frame
{ 
	We'll see if we have time for that last one. If we don't get to it, it will be included as an optional set of slides. 
}

\subsection{What and How?}

\frame
{ 
	First, we'll start off with some exercises to illustrate the point, proving free theorems and basic facts about types (e.g. reasoning through why \red{$\forall a. a \to a$} has precisely 1 inhabitant), and then draw direct lines with the Lambek correspondence. 
}

\frame
{ 
	Second, we will try to understand the Leibniz perspective and how it informs the Curry-Howard correspondence. 
}

\frame
{ 
	Third, we will begin to build the requisite knowledge for understanding the Yoneda perspective, and build some categorical constructs, and an understanding of representation. 
}

\frame
{ 
	Then, we will tie it all together. 
}

\frame
{ 
	Special thanks to Jon Pretty (\href{https://twitter.com/propensive}{@propensive})
}

\section{Some Type Arithmetic}

\frame
{
	Let's do a quick warm up 
}

\subsection{Ground Rules}

\frame
{ 
	We will use the following conventions: 
	
	\begin{itemize}
		\item the type \red{()} with precisely 1 inhabitant (called \blue{Unit}) will be denoted by the number \red{1}. 
		\item the type with no inhabitants, \blue{Nothing} or \red{undefined} will be denoted \red{0}. 
		\item Function types will be denoted by \red{$a \to b$} or by the notation \red{$b^a$}. 
		\item Universal quantification is given by \red{forall} or \red{$\forall$}.
	\end{itemize}
}

\frame
{
	\begin{itemize}
		\item Our calculus assumes that we have all finite coproducts and will denote them by the plus sign \red{+}. In Scala and Haskell, the binary coproduct is \red{Either}.
		\item Our calculus assumes that we have all finite products, and we will denote they by the times sign \red{*}. In Scala and Haskell, the binary product is \red{(,)} or (\red{(,)} is an alias for \blue{Tuple2} in Scala.
	\end{itemize}
}

\frame
{ 
	Proposition: Suppose we were to treat a polymorphic like your standard high school arithmetic. What would this arithmetic look like? 
}

\subsection{Addition}

\frame
{ 
	Does \mred{1 + 1 = 2}? How can we tell? 
}

\begin{frame}[fragile]
	By simple counting. Axiomatically, \red{1} is the type with \red{1} inhabitant, so
	\begin{minted}{haskell}
	
	1 + 1  
	= () + ()
	= Either () ()
	\end{minted}
\end{frame}

\frame
{ 
	By inspection, we see that \red{Either} has precisely two cases - \blue{Left} and \blue{Right} containing a single type. Thus, two cases, with one inhabitant in each case. Therefore we have 2 inhabitants!
}

\frame
{ 
	How about \mred{Maybe} containing values of type \mblue{\bm{Bool} = \{ \top, \bot\}}? 
}

\begin{frame}[fragile]

Again, we can apply some natural deduction, nothing that \mblue{\bm{Bool}} has precisely two inhabitants
	\begin{minted}{haskell}
	
	Maybe Bool
	= Nothing + Just (True + False)
	= 1 + 1 + 1
	= 3
	\end{minted}
\end{frame}

\frame
{
	Simple right? Lets move on.
}

\subsection{Multiplication}

\frame
{

	Multiplication operates in the similar way. Using products, we can count \mred{\bm{Bool} * \bm{Bool}}. We have precisely four cases: 
	
	\begin{itemize}	
		\item \mblue{(True, True)}
		\item \mblue{(True, False)}
		\item \mblue{(False, True)}
		\item \mblue{(False, False)} 
	\end{itemize}
}

\frame
{
	This works out as well! How about exponents?
}

\subsection{Exponentiation}


\subsection{() and ⊥}

\end{document}
